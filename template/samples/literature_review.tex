

Live experimentation of physical processes is often not practical. The costs of real life experimentation can be prohibitively high, and can require significant time in order to fully ascertain the long-term trends. There is also a risk that changes can have detrimental impacts and therefore can lead to only minor tweaks being made ~\cite{Forshaw2016}. These factors are particularly true for electricity markets, where decisions can have long term impacts, as generation plants often having a long lifetime compounding decisions. Simulation, however, can be used for rapidly testing and prototyping ideas. The simulation is parametrised by real world data and phenomena. Through simulation, the user is able to assess the likelihoods of outcomes under certain scenarios and parameters \cite{Law:603360}.



%\begin{table*}[]
%%	\begin{tabular}{|l|c|c|c|c|c|}
%	\begin{tabular}{lccccc} \toprule
%		\multicolumn{1}{c}{\textbf{Tool name}} & \textbf{Open Source} & \textbf{Long-Term Investment} & \textbf{Market} & \textbf{Stochastic Inputs} & \textbf{Country Generalisability} \\ \midrule
%		SEPIA \cite{Harp2000}  & \checkmark           & $\times$                             & \checkmark      & Demand                     & \checkmark                        \\ 
%		EMCAS ~\cite{Conzelmann}   & $\times$                    & \checkmark                    & \checkmark      & Outages                    & \checkmark                        \\ 
%		NEMSIM ~\cite{Batten2006}  & ?              & \checkmark                    & \checkmark      & $\times$                          & $\times$                                 \\ 
%		AMES  ~\cite{Sun2007} & \checkmark           & $\times$                             & Day-ahead       & $\times$                          & $\times$                                 \\ 
%		PowerACE ~\cite{Rothengatter2007} & $\times$                    & \checkmark                    & \checkmark      & Outages/Demand             & \checkmark                        \\ 
%		MACSEM  ~\cite{Praca2003}  & ?              & $\times$                             & \checkmark      & $\times$                          & \checkmark                        \\ 
%		GAPEX  ~\cite{Cincotti2013} & ?              & $\times$                             & Day-ahead       & $\times$                          & \checkmark                        \\ 
%		EMLab ~\cite{Chappin2017}  & \checkmark           & \checkmark                    & Futures         & $\times$                          & \checkmark                        \\ 
%		ElecSIM                                  & \checkmark           & \checkmark                    & Futures         & \checkmark                 & \checkmark                        \\ \hline
%	\end{tabular}
%	\caption{Features of electricity market agent based model tools.}
%	\label{table:tool_comparison}
%\end{table*}


% {\color{red}Electricity energy policy modelling is an example where simulation can be used. Real-life experimentation of energy policy is not always feasible due to the long times required to observe results and high risks associated with setting a sub-optimal policy which could radically alter business models and lead to blackouts in electricity supply. Decisions can have long-term impacts, such as producing an electricity market with many expensive and highly polluting coal power plants, they may have ramp-up times, which is the time it takes for a generator to increase electricity production, that are not suitable to accommodate the intermittent electrical flow of renewables. Intermittent electrical flow is where sources of electricity exhibit uncontrolled increases or decreases in output, which is often the case for renewables such as wind, solar, wave and tidal \cite{Challenges2016}. A number of different simulations and computer models have been used to aid policy makers and energy market developers in coming to informed conclusions:}

Energy models can typically be classified as top-down macro-economic models or bottom-up techno-economic models~\cite{Bohringer1998}. Top-down models typically focus on behavioural realism with a focus on macro-economic metrics. They are useful for studying economy-wide responses to policies ~\cite{Hall2016}, for example MARKAL-MACRO \cite{Fishbone1981} and LEAP \cite{Heaps2016}. Bottom-up models represent the energy sector in detail, and are written as mathematical programming problems~\cite{Gargiulo2013}. They detail technology explicitly, and can include cost and emissions implications~\cite{Hall2016}.

It is possible to further categorise bottom-up models into optimisation and simulation models. Optimisation energy models minimise costs or maximise welfare, defined as the material and physical well-being of people, from the perspective of a central planner ~\cite{Keles2017}. A use-case would be a government that wants cheap, reliable and a low-carbon electricity supply by a specified date. An optimisation model would find the optimal mix of generators to meet this whilst taking into account constraints such as space, resources and demand. Examples of optimisation models are MARKAL/TIMES~\cite{Fishbone1981} and MESSAGE~\cite{Schrattenholzer1981}. MARKAL is possibly the most widely used general purpose energy systems model~\cite{Pfenninger2014}.

However, electricity market liberalisation in many western democracies has changed the framework conditions. Centralised, monopolistic, decision making entities have given way to multiple heterogeneous agents acting for their own best interest~\cite{Most2010}. Policy options must therefore be used to encourage changes to attain a desired outcome, such as subsidies or taxes. It is proposed that these complex agents are modelled using ABMs due to their non-deterministic nature. 

% {\color{red}Agent-based simulation for electricity markets has received increasing attention in recent years. }

% {\color{red}There are numerous different mechanisms/markets for GenCos to sell electricity. These mechanisms can largely be split into pool markets and bilateral contracts. A pool market is a market in which bids and offers use supply and demand principles to set the price. Pool markets operators typically provide the function of matching buyers and sellers. Bilateral contracts are typically longer term markets where a GenCo will sell electricity to a demand company based on long-term contracts. }

% {\color{red} These mechanisms can be divided further into day-ahead markets and futures markets. Where day-ahead markets are markets where a buyer assesses how much energy it will need to meet demand the next day, and how much it is willing to pay for this volume, hour by hour \cite{nordpool_20192}. The seller (GenCo), will decide how much electricity it can deliver and at what price, hour by hour. Futures markets is where electricity is traded as a commodity, where electricity is bought at a certain price, at either high or low demand at a point in the future.}



\begin{table}[]
	%	\begin{tabular}{|l|c|c|c|c|c|}
	\begin{tabular}{M{0.9cm}M{1cm}M{1cm}M{1cm}M{1.2cm}M{1cm}} \toprule
		\multicolumn{1}{c}{\textbf{Tool name}} & \textbf{Open Source} & \textbf{Long-Term Investment} & \textbf{Market} & \textbf{Stochastic Inputs} & \textbf{Country Generalisability} \\ \midrule
		SEPIA \cite{Harp2000}  & \checkmark           & $\times$                             & \checkmark      & Demand                     & \checkmark                        \\ 
		EMCAS ~\cite{Conzelmann}   & $\times$                    & \checkmark                    & \checkmark      & Outages                    & \checkmark                        \\ 
		NEMSIM ~\cite{Batten2006}  & ?              & \checkmark                    & \checkmark      & $\times$                          & $\times$                                 \\ 
		AMES  ~\cite{Sun2007} & \checkmark           & $\times$                             & Day-ahead       & $\times$                          & $\times$                                 \\ 
		GAPEX  ~\cite{Cincotti2013} & ?              & $\times$                             & Day-ahead       & $\times$                          & \checkmark                        \\ 
		PowerACE\\\cite{Rothengatter2007} & $\times$                    & \checkmark                    & \checkmark      & Outages Demand             & \checkmark                        \\ 

		EMLab ~\cite{Chappin2017}  & \checkmark           & \checkmark                    & Futures         & $\times$                          & \checkmark                        \\ 
		MACSEM  ~\cite{Praca2003}  & ?              & $\times$                             & \checkmark      & $\times$                          & \checkmark                        \\ 
		ElecSIM                                  & \checkmark           & \checkmark                    & Futures         & \checkmark                 & \checkmark                        \\ \hline
	\end{tabular}
	\caption{Features of electricity market ABM tools.}
	\label{table:tool_comparison}
	\vskip -1.25cm
\end{table}


A number of simulation tools have emerged which model electricity markets: SEPIA~\cite{Harp2000} EMCAS~\cite{Conzelmann}, NEMSIM~\cite{Batten2006}, AMES~\cite{Sun2007}, PowerACE~\cite{Rothengatter2007}, MACSEM ~\cite{Praca2003}, GAPEX~\cite{Cincotti2013} and  EMLab~\cite{Chappin2017}. By referring to Table \ref{table:tool_comparison}, it can be seen that these do not suit the needs of an open source, long-term market model. We will demonstrate that stochasticity of parameters is also required to increase realism.

Table \ref{table:tool_comparison} is made up of six columns. Tool name, whether the tool is open source or not, whether they model long-term investment in electricity infrastructure such as power plants and what markets they model. We determine how the stochasticity of real life is modelled in each tool. Finally, we determine whether the model is generalisable to different countries or whether it is specific to a particular country's regulatory structure. 


An open source toolkit is important for reproducibility, transparency and lowering barriers to entry. It enables users to expand the model to their requirements and respective country. The modelling of long-term investment enables scenarios to emerge, and enable users to model investment behaviour. We demonstrate that stochasticity improves results.

SEPIA \cite{Harp2000} is a discrete event ABM which utilises Q-learning to model the bids made by GenCos. SEPIA models plants as being always on, and does not have an independent system operator (ISO) which in an electricity market, is an independent non-profit organization for coordinating and controlling of regular operations of the electric power system and market \cite{Zhou2007} SEPIA does not model a spot market, instead focusing on bilateral contracts. As opposed to this, ElecSIM has been designed with a merit-order, spot market in mind. As shown in Table \ref{table:tool_comparison}, SEPIA does not include a long-term investment mechanism. 

EMCAS ~\cite{Conzelmann} is a closed source ABM. EMCAS investigates the interactions between physical infrastructures and economic behaviour of agents. However, ElecSIM focuses on the dynamics of the market, and provides a simplified, transparent model of market operation, whilst maintaining robustness of results.

NEMSIM \cite{Grozev2005} is an ABM that represents Australia's National Electricity Market (NEM). Participants are able to grow and change over time using learning algorithms. NEMSIM is non-generalisable to other electricity markets, unlike ElecSIM.

AMES ~\cite{Sun2007} is an ABM specific to the US Wholesale Power Market Platform and therefore not generalizable for other countries. GAPEX \cite{Cincotti2013} is an ABM framework for modelling and simulating power exchanges . GAPEX utilises an enhanced version of the reinforcement technique Roth-Erev \cite{RothAE1995} to consider the presence of affine total cost functions. However, neither of these model the long-term dynamics for which ElecSIM is designed.



PowerACE ~\cite{Rothengatter2007} is also a closed source ABM of electricity markets that integrates short-term perspectives of daily electricity trading and long-term investment decisions. PowerACE models the spot market, forward market and a carbon market. Similarly to ElecSIM, PowerACE initialises GenCos with each of their power plants. However, as can be seen in Table \ref{table:tool_comparison} unlike ElecSIM, PowerACE does not take into account stochasticity of price risks in electricity markets ~\cite{Most2010}.

EMLab ~\cite{Chappin2017} is an open-source ABM toolkit for the electricity market. Like PowerACE, EMLab models an endogenous carbon market, however, they both differ from ElecSIM by not taking into account stochasticity in the electricity markets, such as in outages, fuel prices and operating costs. However, after correspondence with the authors, we were unable to run the current version of EMLab.

MACSEM \cite{Praca2003} has been used to probe the effects of market rules and conditions by testing different bidding strategies. MACSEM does not model long term investments or stochastic inputs.


As can be seen from Table \ref{table:tool_comparison} none of the tools fill each of the characteristics we have defined. We therefore propose ElecSIM to contribute an open source, long-term, stochastic investment model. 


