Due to the threat of climate change, a transition from a fossil-fuel based system to one based on zero-carbon is required. However, this is not as simple as instantaneously closing down all fossil fuel energy generation and replacing them with renewable sources -- careful decisions need to be taken to ensure rapid but stable progress. To aid decision makers, we present a new tool, ElecSIM, which is an open-sourced agent-based modelling framework used to examine the effect of policy on long term investment decisions in the electricity sector. ElecSIM allows non-experts to rapidly prototype new ideas, and is developed around a modular framework -- which allows technical experts to add and remove features at will. 

Different techniques to model long term electricity decisions are reviewed and used to motivate why agent-based models will become an important strategic tool for policy. We motivate why an open-source toolkit is required for long-term electricity planning.

Actual electricity prices are compared with our model and we demonstrate that the modelling of stochasticity in the system improves performance by $52.5\%$. Further, using ElecSIM we demonstrate the effect of a carbon tax to encourage a low-carbon electricity supply. We show how a \textsterling40 ($\$50$) per tonne of carbon emitted would lead to 70\% renewable electricity energy market by 2050. \vphantom{{\color{red}An interesting note, however, is that starting with a low carbon tax and slowly increasing this by the year 2050 provides similar benefits to a lower, but consistent tax in the long run, due to the high capital costs and long operating periods of generators. This has the benefits of reducing costs as well as providing certainty to investors.}}
