

Liberalised electricity markets with many heterogenous players are suited to be modelled with ABMs.

ABMs are able to model imperfect information as well as heterogeneous actors. ElecSIM models imperfect information through forecasting of electricity demand and future fuel and electricity prices. This leads to agents taking risk on their investments, and more realistically model market conditions.

We demonstrated that increasing carbon tax can lead to an increase in investment of low-carbon technologies. We showed that early decisions have a long-term impact on the energy mix. 

Our future work includes comparing agent-learning techniques, using multi-agent reinforcement learning algorithms and artificial intelligence to allow agents to learn in a non-static environment. We propose the integration of a higher temporal and spatial resolution to model changes in daily demand, as well as capacity factors by region, and transmission effects. This will allow us to model that demand is met at all times and not just on average. We propose the modelling of collusion between GenCos.

%\begin{itemize}
%	\item Requirement for agent based models based on imperfect information, liberalised energy markets
%	\item Requirement for low barriers to entry open source model.
%	\item Discuss results
%	\item Future work:
%	\begin{itemize}
%		\item Embedding multi-agent intelligence such as Genetic Algorithms,  Q-learning and dynamic reinforcement learning
%		\item Raise spatial and temporal resolution.
%	\end{itemize}
%\end{itemize}

\FloatBarrier